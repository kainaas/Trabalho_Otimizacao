%!TeX root = relatorio.tex
% █▀▄▀█ ▄▀█ ▀█▀ █░█   █▀█ █▀█ █▀▀ ▄▀█ █▀▄▀█ █▄▄ █░░ █▀▀
% █░▀░█ █▀█ ░█░ █▀█   █▀▀ █▀▄ ██▄ █▀█ █░▀░█ █▄█ █▄▄ ██▄

%::::::::::::::::::::::::::::::::::::::::::::::::::::::::::::::::::::::::::::::::
%                                   CORE SETUP                                   
%                             Document-Wide Settings                             
%................................................................................
\documentclass[
	% -- opções da classe memoir --
	12pt,				% tamanho da fonte
	openany,			% capítulos não pulam folhas
	oneside,			% <--- ALTERADO: 'oneside' mantêm a margem fixa (ideal para digital)
	a4paper,			% tamanho do papel. 
	english,			% idioma adicional para hifenização
	brazil				% o último idioma é o principal do documento
	]{abntex2}

\pagestyle{plain}

\usepackage[utf8]{inputenc}		% Input encoding
\usepackage[T1]{fontenc}		% Output enconding
\usepackage[ 
	left=3cm,
	top=3cm,
	right=2cm,
	bottom=2cm
]{geometry}				        % Margens ABNT

\usepackage[brazil]{babel}		% Configurações de linguagem
\addto\extrasbrazil{%
  \def\sectionautorefname{Seção}%
  \def\subsectionautorefname{Subseção}%
  \def\subsubsectionautorefname{Subsubseção}%
  \def\figureautorefname{Figura}%
  \def\tableautorefname{Tabela}%
  \def\equationautorefname{Equação}%
  \def\lemmaautorefname{Lema}%
  \def\theoremautorefname{Teorema}%
  \def\propositionautorefname{Proposição}%
  \def\corollaryautorefname{Corolário}%
  \def\definitionautorefname{Definição}%
  \def\problemauorefname{Problema}%
  \floatname{algorithm}{Algoritmo}
  \algrenewcommand\algorithmicrequire{\textbf{Entrada:}}
  \algrenewcommand\algorithmicensure{\textbf{Saída:}}
} 
\usepackage{csquotes}			

%::::::::::::::::::::::::::::::::::::::::::::::::::::::::::::::::::::::::::::::::
%                                    STYLING                                     
%                        Alter Visual Appearance of Text                         
%................................................................................
\usepackage{setspace}		% Espaçamento entre linhas
\usepackage[sc]{mathpazo}	% Fonte Palatino
\linespread{1.05}		    % Ajuste para Palatino
\usepackage[dvipsnames]{xcolor}	% Cores
\usepackage{moresize}		% Tamanhos de fonte extras
\usepackage{enumerate}		% Listas
\usepackage{multicol}		% Múltiplas colunas
\usepackage[normalem]{ulem}	% Sublinhados
\usepackage{microtype}      % <--- IMPORTANTE: Melhora o espaçamento entre palavras

%::::::::::::::::::::::::::::::::::::::::::::::::::::::::::::::::::::::::::::::::
%                                     FLOATS                                     
%               Manage Floating Content (e.g.: figures and tables)               
%................................................................................
\usepackage{subcaption}  
\usepackage{booktabs}    
\usepackage{float}	 
\usepackage{algorithm}     
\usepackage{algpseudocode}  

%:::::::::::::::::::::::::::::::::::::::::::::::::::::::::::::::::::::::::::::::::
%                                    GRAPHICS                                    
%                             Generate/Import Images                              
%.................................................................................
\usepackage{tikz}		
\usetikzlibrary{arrows.meta}	
\usepackage{pgfplots}		
\pgfplotsset{compat=1.18}	
\usepackage{graphicx}		

%:::::::::::::::::::::::::::::::::::::::::::::::::::::::::::::::::::::::::::::::::
%                                 MATH PACKAGES                                  
%.................................................................................
\usepackage{amssymb} 	
\usepackage{amsmath}	
\usepackage{mathtools}  
\usepackage{amsthm}    	
\usepackage{xfrac}	
\usepackage{cancel}	
\usepackage{bbm}	
\usepackage[		
  scr=boondox,	
  cal=esstix	
]{mathalpha}

%::::::::::::::::::::::::::::::::::::::::::::::::::::::::::::::::::::::::::::::::::
%                            CUSTOM THEOREM ENVIRONMENTS                            
%..................................................................................
\colorlet{lmm}{DarkOrchid}	
\colorlet{thm}{Tan}		
\colorlet{prop}{Maroon}		
\colorlet{crl}{RawSienna}	
\colorlet{eg}{RoyalPurple}	
\colorlet{ex}{RoyalPurple}	
\colorlet{obs}{MidnightBlue}	
\colorlet{note}{PineGreen}	

% ESTILOS DE TEOREMA
\newtheoremstyle{number-name}{4mm}{4mm}{\itshape}{}{\bfseries}{}{.3em}{\thmnumber{#2}\,\thmname{#1}}
\newtheoremstyle{name-it_number}{4mm}{4mm}{\itshape}{}{\bfseries}{}{.3em}{\thmname{#1}\,{\itshape\thmnumber{#2}}}
\newtheoremstyle{no_number}{4mm}{4mm}{}{}{\bfseries}{}{.3em}{}
\newtheoremstyle{no_number-it_body}{4mm}{4mm}{\itshape}{}{\bfseries}{}{.3em}{}

% DEFINIÇÃO DOS AMBIENTES
\theoremstyle{number-name}
  \newtheorem{lemma}{Lema:}[section]
  \newtheorem{theorem}[lemma]{Teorema:}
  \newtheorem{proposition}[lemma]{Proposição:}
  \newtheorem{corollary}[lemma]{Corolário:}
  \newtheorem{definition}[lemma]{Definição:}

\theoremstyle{name-it_number}
  \newtheorem{exercise}{\textcolor{ex}{Exercício}}[section]
  \newtheorem{example}[exercise]{\textcolor{eg}{Exemplo}}
  \newtheorem{problem}[exercise]{Problema}

\theoremstyle{no_number}
  \newtheorem*{observation*}{Observação:}
  \newtheorem*{solution}{Solução:}
  \newtheorem*{proofT}{Demonstração:}
  \newtheorem*{proofL}{Demonstração:}
  \newtheorem*{proofC}{Demonstração:}
  \newtheorem*{proofP}{Demonstração:}
  \newtheorem*{proofE}{Demonstração:}
  \newtheorem*{proofO}{Demonstração:}

\theoremstyle{no_number-it_body}
  \newtheorem*{notation*}{\color{note}Notação:}

\renewcommand\qedsymbol{\scriptsize $\blacksquare$}

% COMANDOS MATEMÁTICOS
\newcommand{\e}{\mathscr{e}}		
\newcommand{\longto}{\longrightarrow}	
\newcommand*{\defeq}{\mathrel{\vcenter{\baselineskip0.5ex \lineskiplimit0pt \hbox{\scriptsize.}\hbox{\scriptsize.}}} =}					
\newcommand{\subnormal}{\triangleleft}	
\newcommand{\supnormal}{\triangleright}	
\newcommand{\limplies}{\rightarrow}	
\newcommand{\liff}{\leftrightarrow}	
\newcommand{\nand}{\uparrow}		
\newcommand{\nor}{\downarrow}		
\newcommand{\xor}{\veebar}		

%:::::::::::::::::::::::::::::::::::::::::::::::::::::::::::::::::::::::::::::::::
%                                  BIBLIOGRAPHY                                   
%.................................................................................
\usepackage[
	backend=biber,	
	style=abnt,	
	sorting=nyt	
]{biblatex}

%:::::::::::::::::::::::::::::::::::::::::::::::::::::::::::::::::::::::::::::::::
%                                  FINALIZATION                                   
%.................................................................................
\usepackage{xurl}           % <--- ADICIONADO: Quebra URLs em qualquer lugar (corrige links grandes)
\usepackage{hyperref}		% Links clicáveis
\hypersetup{
	colorlinks=true,	
	allcolors=NavyBlue,	
}

\newcommand{\customref}[2]{%	
	\hyperref[#2]{#1}%
}

% <--- ADICIONADO: Impede que o LaTeX estique o texto verticalmente para preencher a página
\raggedbottom