% █▀▄▀█ ▄▀█ ▀█▀ █░█   █▀█ █▀█ █▀▀ ▄▀█ █▀▄▀█ █▄▄ █░░ █▀▀
% █░▀░█ █▀█ ░█░ █▀█   █▀▀ █▀▄ ██▄ █▀█ █░▀░█ █▄█ █▄▄ ██▄

%::::::::::::::::::::::::::::::::::::::::::::::::::::::::::::::::::::::::::::::::
%                                   CORE SETUP                                   
%                             Document-Wide Settings                             
%................................................................................
\documentclass[
	% -- opções da classe memoir --
	12pt,				% tamanho da fonte
	openany,			% capítulos não pulam folhas
	twoside,			% para impressão em verso e anverso. Oposto a oneside
	a4paper,			% tamanho do papel. 
	% -- opções da classe abntex2 --
	%chapter=TITLE,		% títulos de capítulos convertidos em letras maiúsculas
	%section=TITLE,		% títulos de seções convertidos em letras maiúsculas
	%subsection=TITLE,	% títulos de subseções convertidos em letras maiúsculas
	%subsubsection=TITLE,% títulos de subsubseções convertidos em letras maiúsculas
	% -- opções do pacote babel --
	english,			% idioma adicional para hifenização
	brazil				% o último idioma é o principal do documento
	]{abntex2}


\usepackage[utf8]{inputenc}		% Input encoding: Assumes UTF-8 text editor
\usepackage[T1]{fontenc}		% Output enconding: 8-bit, for better glyph support
\usepackage[ %ABNT margins
	left=3cm,
	top=3cm,
	right=2cm,
	bottom=2cm
]
{geometry}				% Set page margins (ABNT standard)

\usepackage[brazil]{babel}		% Language-specific settings
\addto\extrasbrazil{%
  \def\sectionautorefname{Seção}%
  \def\subsectionautorefname{Subseção}%
  \def\subsubsectionautorefname{Subsubseção}%
  \def\figureautorefname{Figura}%
  \def\tableautorefname{Tabela}%
  \def\equationautorefname{Equação}%
  % Custom amsthm environments
  \def\lemmaautorefname{Lema}%
  \def\theoremautorefname{Teorema}%
  \def\propositionautorefname{Proposição}%
  \def\corollaryautorefname{Corolário}%
  \def\definitionautorefname{Definição}%
  % Algorithm environment from algpseudocode
  \floatname{algorithm}{Algoritmo}
  \algrenewcommand\algorithmicrequire{\textbf{Entrada:}}
  \algrenewcommand\algorithmicensure{\textbf{Saída:}}
} %
\usepackage{csquotes}			% Context-sensitive quoting
					% 'csquotes' works in tandem with babel
					% 'csquotes' IS A 'biblatex' DEPENDENCY


%::::::::::::::::::::::::::::::::::::::::::::::::::::::::::::::::::::::::::::::::
%                                    STYLING                                     
%                        Alter Visual Appearance of Text                         
%................................................................................
\usepackage{setspace}		% Allows for custom line spacing
\usepackage[sc]{mathpazo}	% Set the font to Palatino
\linespread{1.05}		% Recommended spacing adjutment for Palatino
\usepackage[dvipsnames]{xcolor}	% Access to 68 named colors
\usepackage{moresize}		% Additional font sizes
\usepackage{enumerate}		% Customize item style in 'enumerate'
\usepackage{multicol}		% Environment with multiple columns of text
\usepackage[normalem]{ulem}	% More underline styles (and aligns bottoms of multiple underlines)
				% 'normalem' prevents underlines from altering \emph

%::::::::::::::::::::::::::::::::::::::::::::::::::::::::::::::::::::::::::::::::
%                                     FLOATS                                     
%               Manage Floating Content (e.g.: figures and tables)               
%................................................................................
\usepackage{subcaption}  % Format multiple images(tables) in one figure(table) environment
\usepackage{booktabs}    % Format tables following scientific convention
\usepackage{float}	 % Provides [H] 'Here' float placement specifier (use with caution)
\usepackage{algorithm}     % Float environment for algorithms
\usepackage{algpseudocode}  % Pseudocode commands for 'algorithm' environment


%:::::::::::::::::::::::::::::::::::::::::::::::::::::::::::::::::::::::::::::::::
%                                    GRAPHICS                                    
%                             Generate/Import Images                              
%.................................................................................
% NATIVE PROGRAMMATIC DIAGRAMS & PLOTS
\usepackage{tikz}		% Create vector graphics programmatically
\usetikzlibrary{arrows.meta}	% Additional arrow styles
\usepackage{pgfplots}		% Tool for creating function plots and charts
\pgfplotsset{compat=1.18}	% Enforce version 1.18 for consistency

% EXTERNAL IMAGE MANAGEMENT
\usepackage{graphicx}		% LaTeX standard for including external image files


% █▀▄▀█ ▄▀█ ▀█▀ █░█
% █░▀░█ █▀█ ░█░ █▀█
%:::::::::::::::::::::::::::::::::::::::::::::::::::::::::::::::::::::::::::::::::
%                                 MATH PACKAGES                                  
%.................................................................................
% AMS (American Mathematical Society)
\usepackage{amssymb} 	% AMS's package for mathematical symbols
\usepackage{amsmath}	% AMS's package for mathematical typesetting
\usepackage{mathtools}  % Extends amsmath's functionalities (i.e.: a superset)
\usepackage{amsthm}    	% AMS's package for creating theorem-like environments

% NICHE MATH PACKAGES
\usepackage{xfrac}	% Slanted fractions (\sfrac)
\usepackage{cancel}	% Strike-through for symbols (\cancel and \cancelto)
\usepackage{bbm}	% Shorthand solution for blackboard boldfaced 1 (use \mathbbm{1})
\usepackage[		% Control over fonts used in \mathbb, \mathcal, \mathfrak & \mathscr
  scr=boondox,	% 'boondox' is a script font that suports lowercase letters
  cal=esstix	% 'esstix' is a calligraphic font that supports lowercase, but not bold letters
]{mathalpha}

%::::::::::::::::::::::::::::::::::::::::::::::::::::::::::::::::::::::::::::::::::
%                           CUSTOM THEOREM ENVIRONMENTS                            
%                                   (Color Coded)                                  
%..................................................................................
% 1. DEFINE COLOR CODE FOR EACH THEOREM-LIKE ENVIRONMENT (makes change of color choice effortless)
\colorlet{lmm}{DarkOrchid}	% For lemma env.
\colorlet{thm}{Tan}		% For theorem env.
\colorlet{prop}{Maroon}		% For proposition env.
\colorlet{crl}{RawSienna}	% For corollary env.
\colorlet{def}{OliveGreen}	% For definition env.
\colorlet{eg}{RoyalPurple}	% For example env.
\colorlet{ex}{RoyalPurple}	% For exercise env.
\colorlet{obs}{MidnightBlue}	% For remark env.
\colorlet{note}{PineGreen}	% For notation remark env.

%. . . . . . . . . . . . . . . . . . . . . . . . . . . . . . . . . . . . . . . . . 
% 2. DEFINE CUSTOM THEOREM STYLES
% 2.1. STYLE I - TAUTOLOGIES - FIRST NUMBER, THEN NAME
\newtheoremstyle{number-name}
{4mm}					  % Space above env.
{4mm}					  % Space below env.
{\itshape}				  % Body font
{}					  % Indent amount
{\bfseries}				  % Head font
{}					  % Punctuation after head
{.3em}					  % Space after head
{\thmnumber{#2}\,\thmname{#1}}		  % Head spec

% 2.2. STYLE II - EXAMPLES & EXERCISES - FIRST NAME, THEN ITALIC NUMBER 
\newtheoremstyle{name-it_number}
{4mm}						% Space above env.
{4mm}						% Space below env.
{\itshape}					% Body font
{}						% Indent amount
{\bfseries}					% Head font
{}						% Punctuation after head
{.3em}						% Space after head
{\thmname{#1}\,{\itshape\thmnumber{#2}}}	% Head spec

% 2.3. STYLE III - REMARKS, SOLUTIONS & PROOFS - NO NUMBERING (should refer to a numbered env.)
\newtheoremstyle{no_number}
{4mm}						% Space above env.
{4mm}						% Space below env.
{}						% Body font
{}						% Indent amount
{\bfseries}					% Head font
{}						% Punctuation after head
{.3em}						% Space after head
{}						% Head spec

% 2.4. STYLE IV - NOTATION REMARKS - NO NUMBERING (should refer to a numbered env.) AND ITALIC BODY 
\newtheoremstyle{no_number-it_body}
{4mm}						% Space above
{4mm}						% Space below
{\itshape}					% Body font
{}						% Indent amount
{\bfseries}					% Head font
{}						% Punctuation after head
{.3em}						% Space after head
{}						% Head spec

%. . . . . . . . . . . . . . . . . . . . . . . . . . . . . . . . . . . . . . . . .
% 3. APPLY THEOREM STYLES AND DEFINE RESPECTIVE ENVIRONMENTS
% 3.1. STYLE I - TAUTOLOGIES - FIRST NUMBER, THEN NAME
\theoremstyle{number-name}
  \newtheorem{lemma}{Lema:}[section]
  \newtheorem{theorem}[lemma]{Teorema:}
  \newtheorem{proposition}[lemma]{Proposição:}
  \newtheorem{corollary}[lemma]{Corolário:}
  \newtheorem{definition}[lemma]{Definição:}

% 3.2. STYLE II - EXAMPLES & EXERCISES - FIRST NAME, THEN ITALIC NUMBER 
\theoremstyle{name-it_number}
  \newtheorem{exercise}{\textcolor{ex}{Exercício}}[section]
  \newtheorem{example}[exercise]{\textcolor{eg}{Exemplo}}

% 3.3. STYLE III - REMARKS, SOLUTIONS & PROOFS - NO NUMBERING
\theoremstyle{no_number}
  % Remark
  \newtheorem*{observation*}{Observação:}
  % Solution
  \newtheorem*{solution}{Solução:}
  % Proofs
  \newtheorem*{proofT}{Demonstração:}
  \newtheorem*{proofL}{Demonstração:}
  \newtheorem*{proofC}{Demonstração:}
  \newtheorem*{proofP}{Demonstração:}
  \newtheorem*{proofE}{Demonstração:}
  \newtheorem*{proofO}{Demonstração:}

% 3.4. STYLE IV - NOTATION REMARKS - NO NUMBERING AND ITALIC BODY 
\theoremstyle{no_number-it_body}
  \newtheorem*{notation*}{\color{note}Notação:}

%. . . . . . . . . . . . . . . . . . . . . . . . . . . . . . . . . . . . . . . . .
% EXTRA: CUSTOM QED SYMBOL (a small black square)
\renewcommand\qedsymbol{\scriptsize $\blacksquare$}


%:::::::::::::::::::::::::::::::::::::::::::::::::::::::::::::::::::::::::::::::::
%                              CUSTOM MATH COMMANDS                               
%.................................................................................
% CONSTANTS
\newcommand{\e}{\mathscr{e}}		% Special notation for Euler's number

% RELATIONS
\newcommand{\longto}{\longrightarrow}	% Alias to note function domain/codomain (f: A --> B)
\newcommand*{\defeq}{\mathrel{\vcenter{\baselineskip0.5ex \lineskiplimit0pt
\hbox{\scriptsize.}\hbox{\scriptsize.}}}%
=}					% Compact colon-equals
\newcommand{\subnormal}{\triangleleft}	% Normal subgroup (left)
\newcommand{\supnormal}{\triangleright}	% Normal subgroup (right)

% LOGIC
\newcommand{\limplies}{\rightarrow}	% Conditional connector
\newcommand{\liff}{\leftrightarrow}	% Biconditional connector
\newcommand{\nand}{\uparrow}		% NAND connector
\newcommand{\nor}{\downarrow}		% NOR connector
\newcommand{\xor}{\veebar}		% XOR connector


% LOADED LAST TO AVOID DEPENDENCY ERRORS
%:::::::::::::::::::::::::::::::::::::::::::::::::::::::::::::::::::::::::::::::::
%                                  BIBLIOGRAPHY                                   
%                                Manage References                                
%.................................................................................
\usepackage[
	backend=biber,	% Use Biber backend (more powerful than BibTex)
	style=abnt,	% ABNT style
	sorting=nyt	% 'nyt' stands for name, year, title (ABNT standard)
]{biblatex}


%:::::::::::::::::::::::::::::::::::::::::::::::::::::::::::::::::::::::::::::::::
%                                  FINALIZATION                                   
%                 Packages That Patch Others. MUST BE LOADED LAST                 
%.................................................................................
\usepackage{hyperref}		% Add hyperlinks to references, ToC and URLs
\hypersetup{
	colorlinks=true,	% true: colored links; false: boxed links
	allcolors=NavyBlue,	% Set all link types to a single color
}

\newcommand{\customref}[2]{%	% Simple alias for custom link text
	\hyperref[#2]{#1}%
}