%%%%%%%%%%%%%%%%%%%%%%%%%%%%%%%%%%%% PREAMBLE %%%%%%%%%%%%%%%%%%%%%%%%%%%%%%%%%%%%
% 1. Load 'Universal' Toolkit (Project-Agnostic)
%!TeX root = relatorio.tex
% █▀▄▀█ ▄▀█ ▀█▀ █░█   █▀█ █▀█ █▀▀ ▄▀█ █▀▄▀█ █▄▄ █░░ █▀▀
% █░▀░█ █▀█ ░█░ █▀█   █▀▀ █▀▄ ██▄ █▀█ █░▀░█ █▄█ █▄▄ ██▄

%::::::::::::::::::::::::::::::::::::::::::::::::::::::::::::::::::::::::::::::::
%                                   CORE SETUP                                   
%                             Document-Wide Settings                             
%................................................................................
\documentclass[
	% -- opções da classe memoir --
	12pt,				% tamanho da fonte
	openany,			% capítulos não pulam folhas
	oneside,			% <--- ALTERADO: 'oneside' mantêm a margem fixa (ideal para digital)
	a4paper,			% tamanho do papel. 
	english,			% idioma adicional para hifenização
	brazil				% o último idioma é o principal do documento
	]{abntex2}

\pagestyle{plain}

\usepackage[utf8]{inputenc}		% Input encoding
\usepackage[T1]{fontenc}		% Output enconding
\usepackage[ 
	left=3cm,
	top=3cm,
	right=2cm,
	bottom=2cm
]{geometry}				        % Margens ABNT

\usepackage[brazil]{babel}		% Configurações de linguagem
\addto\extrasbrazil{%
  \def\sectionautorefname{Seção}%
  \def\subsectionautorefname{Subseção}%
  \def\subsubsectionautorefname{Subsubseção}%
  \def\figureautorefname{Figura}%
  \def\tableautorefname{Tabela}%
  \def\equationautorefname{Equação}%
  \def\lemmaautorefname{Lema}%
  \def\theoremautorefname{Teorema}%
  \def\propositionautorefname{Proposição}%
  \def\corollaryautorefname{Corolário}%
  \def\definitionautorefname{Definição}%
  \def\problemauorefname{Problema}%
  \floatname{algorithm}{Algoritmo}
  \algrenewcommand\algorithmicrequire{\textbf{Entrada:}}
  \algrenewcommand\algorithmicensure{\textbf{Saída:}}
} 
\usepackage{csquotes}			

%::::::::::::::::::::::::::::::::::::::::::::::::::::::::::::::::::::::::::::::::
%                                    STYLING                                     
%                        Alter Visual Appearance of Text                         
%................................................................................
\usepackage{setspace}		% Espaçamento entre linhas
\usepackage[sc]{mathpazo}	% Fonte Palatino
\linespread{1.05}		    % Ajuste para Palatino
\usepackage[dvipsnames]{xcolor}	% Cores
\usepackage{moresize}		% Tamanhos de fonte extras
\usepackage{enumerate}		% Listas
\usepackage{multicol}		% Múltiplas colunas
\usepackage[normalem]{ulem}	% Sublinhados
\usepackage{microtype}      % <--- IMPORTANTE: Melhora o espaçamento entre palavras

%::::::::::::::::::::::::::::::::::::::::::::::::::::::::::::::::::::::::::::::::
%                                     FLOATS                                     
%               Manage Floating Content (e.g.: figures and tables)               
%................................................................................
\usepackage{subcaption}  
\usepackage{booktabs}    
\usepackage{float}	 
\usepackage{algorithm}     
\usepackage{algpseudocode}  

%:::::::::::::::::::::::::::::::::::::::::::::::::::::::::::::::::::::::::::::::::
%                                    GRAPHICS                                    
%                             Generate/Import Images                              
%.................................................................................
\usepackage{tikz}		
\usetikzlibrary{arrows.meta}	
\usepackage{pgfplots}		
\pgfplotsset{compat=1.18}	
\usepackage{graphicx}		

%:::::::::::::::::::::::::::::::::::::::::::::::::::::::::::::::::::::::::::::::::
%                                 MATH PACKAGES                                  
%.................................................................................
\usepackage{amssymb} 	
\usepackage{amsmath}	
\usepackage{mathtools}  
\usepackage{amsthm}    	
\usepackage{xfrac}	
\usepackage{cancel}	
\usepackage{bbm}	
\usepackage[		
  scr=boondox,	
  cal=esstix	
]{mathalpha}

%::::::::::::::::::::::::::::::::::::::::::::::::::::::::::::::::::::::::::::::::::
%                            CUSTOM THEOREM ENVIRONMENTS                            
%..................................................................................
\colorlet{lmm}{DarkOrchid}	
\colorlet{thm}{Tan}		
\colorlet{prop}{Maroon}		
\colorlet{crl}{RawSienna}	
\colorlet{eg}{RoyalPurple}	
\colorlet{ex}{RoyalPurple}	
\colorlet{obs}{MidnightBlue}	
\colorlet{note}{PineGreen}	

% ESTILOS DE TEOREMA
\newtheoremstyle{number-name}{4mm}{4mm}{\itshape}{}{\bfseries}{}{.3em}{\thmnumber{#2}\,\thmname{#1}}
\newtheoremstyle{name-it_number}{4mm}{4mm}{\itshape}{}{\bfseries}{}{.3em}{\thmname{#1}\,{\itshape\thmnumber{#2}}}
\newtheoremstyle{no_number}{4mm}{4mm}{}{}{\bfseries}{}{.3em}{}
\newtheoremstyle{no_number-it_body}{4mm}{4mm}{\itshape}{}{\bfseries}{}{.3em}{}

% DEFINIÇÃO DOS AMBIENTES
\theoremstyle{number-name}
  \newtheorem{lemma}{Lema:}[section]
  \newtheorem{theorem}[lemma]{Teorema:}
  \newtheorem{proposition}[lemma]{Proposição:}
  \newtheorem{corollary}[lemma]{Corolário:}
  \newtheorem{definition}[lemma]{Definição:}

\theoremstyle{name-it_number}
  \newtheorem{exercise}{\textcolor{ex}{Exercício}}[section]
  \newtheorem{example}[exercise]{\textcolor{eg}{Exemplo}}
  \newtheorem{problem}[exercise]{Problema}

\theoremstyle{no_number}
  \newtheorem*{observation*}{Observação:}
  \newtheorem*{solution}{Solução:}
  \newtheorem*{proofT}{Demonstração:}
  \newtheorem*{proofL}{Demonstração:}
  \newtheorem*{proofC}{Demonstração:}
  \newtheorem*{proofP}{Demonstração:}
  \newtheorem*{proofE}{Demonstração:}
  \newtheorem*{proofO}{Demonstração:}

\theoremstyle{no_number-it_body}
  \newtheorem*{notation*}{\color{note}Notação:}

\renewcommand\qedsymbol{\scriptsize $\blacksquare$}

% COMANDOS MATEMÁTICOS
\newcommand{\e}{\mathscr{e}}		
\newcommand{\longto}{\longrightarrow}	
\newcommand*{\defeq}{\mathrel{\vcenter{\baselineskip0.5ex \lineskiplimit0pt \hbox{\scriptsize.}\hbox{\scriptsize.}}} =}					
\newcommand{\subnormal}{\triangleleft}	
\newcommand{\supnormal}{\triangleright}	
\newcommand{\limplies}{\rightarrow}	
\newcommand{\liff}{\leftrightarrow}	
\newcommand{\nand}{\uparrow}		
\newcommand{\nor}{\downarrow}		
\newcommand{\xor}{\veebar}		

%:::::::::::::::::::::::::::::::::::::::::::::::::::::::::::::::::::::::::::::::::
%                                  BIBLIOGRAPHY                                   
%.................................................................................
\usepackage[
	backend=biber,	
	style=abnt,	
	sorting=nyt	
]{biblatex}

%:::::::::::::::::::::::::::::::::::::::::::::::::::::::::::::::::::::::::::::::::
%                                  FINALIZATION                                   
%.................................................................................
\usepackage{xurl}           % <--- ADICIONADO: Quebra URLs em qualquer lugar (corrige links grandes)
\usepackage{hyperref}		% Links clicáveis
\hypersetup{
	colorlinks=true,	
	allcolors=NavyBlue,	
}

\newcommand{\customref}[2]{%	
	\hyperref[#2]{#1}%
}

% <--- ADICIONADO: Impede que o LaTeX estique o texto verticalmente para preencher a página
\raggedbottom

\addbibresource{Otimizacao-bib-relatorio.bib}
\graphicspath{ {./imagens/} }


\titulo{Aplicação do problema da p-mediana na distrubuição de hospitais na cidade de São Carlos - SP}
\autor{Fabio Kauê Araujo da Silva - 16311045\\
                    Gabriel Inumaru Esteves- 15453487\\
                    Kainã Alves Tureso - 15466391\\
                    MatheusSpinellide Paiva - 14598682\\
                    Pedro Luís Anghievisck - 15656521}
\date{\today}
\local{São Carlos - SP}
\instituicao{USP - Universidade de São Paulo\\
            ICMC - Instituto de Ciências Matemáticas e de Computação}


\begin{document}
% --- TITLE BLOCK ---
    \imprimircapa
    \tableofcontents*

    % --- CONTENT ---
    \chapter{Introdução}

    Afim de proporcionar um melhor atendimento à população de uma determinada cidade, a um determinado serviço, é crucial que o acesso geográfico seja, em alguma métrica, igualitário entre os indivíduos. Uma dessas formas de se medir esse acesso é através da distância entre os indivíduos e os pontos de atendimento. Assim, o problema da p-mediana $(pMP)$ busca, dado um conjunto de pontos (indivíduos) e um número p, selecionar p pontos (locais de atendimento) de forma a minimizar a soma das distâncias entre cada ponto e o ponto mais próximo selecionado.

    Uma das aplicações do $pMP$ é na localização de hospitais em uma cidade, onde o objetivo é minimizar a distância total que os habitantes da cidade precisam percorrer para chegar ao hospital mais próximo. Neste relatório, exploramos a aplicação do problema da p-mediana na distribuição de hospitais na cidade de São Carlos - SP. Utilizando dados geográficos reais, modelamos o problema como um problema de otimização linear e implementamos uma solução computacional utilizando o método Simplex primal.






    \chapter{Objetivos}
    O objetivo desse trabalho é comparar a disposição real de hospitais na cidade de São Carlos-SP com resultados obtidos através da aplicação do $pMP$ como um problema de otimização linear, além de sugirir possíveis novas localizações ótimas (no sentido de minimizar distâncias) para a construção desses estabelecimentos na cidade.





    \chapter{Revisão Bibliográfica}

        \section{Grafos}
        As seguintes definições são dadas em \cite{AURICHI}:

        \begin{definition} \label{def:grafo}
            Um \textbf{grafo} é o par $G=(V,E)$ em que $V$ é um conjunto de pontos denotado \textbf{vértices} e $E\subset V^2$ é o conjunto de \textbf{arestas}.
        \end{definition}

        Informalmente, pode-se definir um grafo como um conjunto de pontos ligados de alguma forma. Um exemplo, que será usado nesse estudo, é a malha viária de uma cidade, na qual as ruas representam as arestas e os pontos são as esquinas. Nesse caso, cada rua tem um tamanho, o que torna o grafo \textbf{ponderado}.
        
        \begin{definition} \label{def:caminho}
            Dizemos que um grafo $G=(V,E)$ não vazio é um \textbf{caminho} se é da forma $V = {x_1, x_2, \dots, x_n}$ e $E = {(x_1,x_2), (x_2,x_3), \dots, (x_{n-1}, x_n)}$
        \end{definition}
        
        As duas definições seguintes serão usadas posteriormente para resultados sobre a possibilidade de transformar o $pMP$ em um problema linear.
        
        \begin{definition} \label{def:conexo}
            Dizemos que um grafo $G=(V,E)$ é \textbf{conexo} se dados dois vértices $u,v \in V$, existe um caminho $A = (P, C)$ tal que $u,v \in P$. 
        \end{definition}
        
        \begin{definition} \label{def:ciclo}
            Dizemos que um grafo é um \textbf{ciclo} se é um caminho e suas extremidades são iguais.
        \end{definition}

        
            \subsection{Algoritmos de distância mínima}

        \section{Problema da p-mediana}

            \subsection{p-mediana como problema de otimização linear}

        \section{Simplex primal}

        
        \section{Interpolação de dados geográficos}

    \chapter{Procedimentos e métodos}
        \section{Modelagem Matemática}    

        \section{Coleta e Tratamento de Dados}

        \section{Implementação Computacional}


    \chapter{Resultados}

    \chapter{Conclusão}


    % --- BIBLIOGRAPHY ---
    \printbibliography
\end{document}